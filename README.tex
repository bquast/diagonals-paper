\documentclass[article]{jss}

\author{
Bastiaan Quast\\The Graduate Institute, Geneva
}
\title{\pkg{diagonals}: Fat Diagonals in R}
\Keywords{matrix, diagonal, block, \proglang{R}}

\Abstract{
The abstract of the article.
}

\Plainauthor{Bastiaan Quast}
\Plaintitle{diagonals: Fat Diagonals in R}
\Shorttitle{\pkg{diagonals}: Fat Diagonals in R}
\Plainkeywords{matrix, diagonal, block, R}

%% publication information
%% \Volume{50}
%% \Issue{9}
%% \Month{June}
%% \Year{2012}
%% \Submitdate{2012-06-04}
%% \Acceptdate{2012-06-04}

\Address{
    Bastiaan Quast\\
  The Graduate Institute, Geneva\\
  Maison de la paix Geneva, Switzerland\\
  E-mail: \href{mailto:bquast@gmail.com}{\nolinkurl{bquast@gmail.com}}\\
  URL: \url{http://qua.st/}
  }

\usepackage{amsmath}

\begin{document}

\section{Introduction}\label{introduction}

Diagonals are an important matrix manipulation. We present the
\texttt{diagonals} R package, which implements functions for dealing
with \textbf{fat diagonals}. Fat diagonals are block matrix-diagonals
that occur when two or more dimensions are mapped along a single edge of
a matrix. For an asymmetric network graph (e.g.~a dyadic social network)
to be mapped to a matrix, we would need each node along each edge of
matrix, however we would also need to map the direction of the tie,
which is an additional dimension. Typically these would be represented
as higher-order arrays (i.e. \textgreater{}= 3). In order to effectively
visualise such arrays, it can be helpful to do so in a matrix. This
could for instance be represented as in the following matrix (where the
\texttt{i} and \texttt{o} suffices represent incoming and outgoing
receptively).

\begin{CodeChunk}
\begin{CodeOutput}
  Ai Ao Bi Bo Ci Co Di Do
A  1  1  0  0  1  0  1  0
B  0  1  1  1  0  1  0  1
C  1  0  0  0  1  1  0  0
D  1  0  1  0  1  1  1  1
\end{CodeOutput}
\end{CodeChunk}

Sometimes the ties of a node to itself are not particularly meaningful
(e.g.~feeling of amiability towards oneself) and can be removed. For a
symmetric network this can simply be done using the function
\texttt{diag()} in R's \texttt{base} package, e.g.

\begin{CodeChunk}
\begin{CodeInput}
sm <- matrix(1, nrow=4, ncol=4)
diag(sm) <- NA
sm
\end{CodeInput}
\begin{CodeOutput}
     [,1] [,2] [,3] [,4]
[1,]   NA    1    1    1
[2,]    1   NA    1    1
[3,]    1    1   NA    1
[4,]    1    1    1   NA
\end{CodeOutput}
\end{CodeChunk}

However, for higher-order matrices this does not work well.

\begin{CodeChunk}
\begin{CodeInput}
diag(m) <- NA
m
\end{CodeInput}
\begin{CodeOutput}
  Ai Ao Bi Bo Ci Co Di Do
A NA  1  0  0  1  0  1  0
B  0 NA  1  1  0  1  0  1
C  1  0 NA  0  1  1  0  0
D  1  0  1 NA  1  1  1  1
\end{CodeOutput}
\end{CodeChunk}

In comes the \texttt{diagonals} package and its workhorse
\texttt{fatdiag()} function. The function is designed to mimic the
behaviour of the \texttt{diag()} as closely as possible, but with then
for \textbf{fat diagonals}.

\begin{CodeChunk}
\begin{CodeInput}
library(diagonals)
\end{CodeInput}
\end{CodeChunk}

\begin{CodeChunk}
\begin{CodeInput}
# the matrix m was restored to its original state
fatdiag(m, steps=4) <- NA
m
\end{CodeInput}
\begin{CodeOutput}
  Ai Ao Bi Bo Ci Co Di Do
A NA NA  0  0  1  0  1  0
B  0  1 NA NA  0  1  0  1
C  1  0  0  0 NA NA  0  0
D  1  0  1  0  1  1 NA NA
\end{CodeOutput}
\end{CodeChunk}

Note that the \texttt{steps} argument defines the number of steps on the
diagonal ladder. Alternatively we could set the \texttt{size} of the
step, more on this later.

\section{Design}\label{design}

The implementation of fat diagonals in the \texttt{diagonals} package is
intended to be as close as possible to the functions dealing with
diagonals included in the \texttt{base} package. As such, the package
includes two functions.

\begin{itemize}
\itemsep1pt\parskip0pt\parsep0pt
\item
  \texttt{fatdiag()}
\item
  \texttt{fatdiag()\textless{}-}
\end{itemize}

These functions offer a very similar syntax to the base functions:

\begin{itemize}
\itemsep1pt\parskip0pt\parsep0pt
\item
  \texttt{diag()}
\item
  \texttt{diag()\textless{}-}
\end{itemize}

With the exception that the fat diagonal functions generally need more
information in terms of the number of \texttt{steps} on the diagonal
ladder, or the \texttt{size} of these steps.

The function \texttt{fatdiag\textless{}-} like its base package
equivalent replaces the (fat) diagonal of its first argument \texttt{x}
with the right side argument \texttt{value}. The \texttt{value} argument
can either be a scalar, in which case it is recycled for the length of
the diagonal, or it can be vector. For the base package function
\texttt{diag()\textless{}-} this vector has be of the same length as the
diagonal (here, the shortest dimension of the matrix), however, the
\texttt{fatdiag()\textless{}-} function will accept any vector that is
of a length that is an integer divisor of the length of the diagonal.
For example, if the length of the diagonal is \texttt{12}, then the
follow lengths for the replacement vector are accepted: \texttt{1},
\texttt{2}, \texttt{3}, \texttt{4}, and \texttt{6}.

The \texttt{fatdiag} function act similar to the \texttt{diag()}
function. Both these functions have two main applications . The first
application is (fat) diagonal extraction, is the first argument
\texttt{x} is a matrix, i.e. \texttt{length(dim(x)) == 2}, then the
function extracts the diagonal matrix and returns it as a vector.

The second application is (fat) diagonal matrix creation. This can be
done in two ways, using a scalar, or using a vector. If a scalar is used
for \texttt{x}, the \texttt{diag()} function returns an identity matrix
\texttt{Ix}, i.e.~a matrix of dimensions \texttt{x} times \texttt{x} is
returned, with \texttt{1} on the diagonal positions and \texttt{0}
elsewhere. The \texttt{fatdiag()} function supports the creation of
non-square matrices (e.g.~using \texttt{size = c(3,2)}) and therefore
uses \texttt{x} as the longest dimension of the matrix, where the other
dimension is determined automatically using the \texttt{size} argument.

\section{Usage}\label{usage}

In the introduction we briefly demonstrate the usage of the
\texttt{fatdiag()} function for assigning a new \texttt{value} to the
fat diagonal. Here we take a closer look at some of the additional
functionality that is implemented.

\begin{CodeChunk}
\begin{CodeInput}
fatdiag(m, size=c(1,2) ) <- 881:888
m
\end{CodeInput}
\begin{CodeOutput}
   Ai  Ao  Bi  Bo  Ci  Co  Di  Do
A 881 882   0   0   1   0   1   0
B   0   1 883 884   0   1   0   1
C   1   0   0   0 885 886   0   0
D   1   0   1   0   1   1 887 888
\end{CodeOutput}
\end{CodeChunk}

So far we have been using the set \texttt{fatdiag()}, i.e.
\texttt{fatdiag()\textless{}-}. However, we can also use the
\texttt{fatdiag()} function either for diagonal extraction, or diagonal
matrix creation.

\begin{CodeChunk}
\begin{CodeInput}
fatdiag(m, steps = 4)
\end{CodeInput}
\begin{CodeOutput}
[1] 881 882 883 884 885 886 887 888
\end{CodeOutput}
\end{CodeChunk}

Fat diagonal matrices can be created using a scalar:

\begin{CodeChunk}
\begin{CodeInput}
fatdiag(9, steps=3)
\end{CodeInput}
\begin{CodeOutput}
      [,1] [,2] [,3] [,4] [,5] [,6] [,7] [,8] [,9]
 [1,]    1    1    1    0    0    0    0    0    0
 [2,]    1    1    1    0    0    0    0    0    0
 [3,]    1    1    1    0    0    0    0    0    0
 [4,]    0    0    0    1    1    1    0    0    0
 [5,]    0    0    0    1    1    1    0    0    0
 [6,]    0    0    0    1    1    1    0    0    0
 [7,]    0    0    0    0    0    0    1    1    1
 [8,]    0    0    0    0    0    0    1    1    1
 [9,]    0    0    0    0    0    0    1    1    1
\end{CodeOutput}
\end{CodeChunk}

or using a vector:

\begin{CodeChunk}
\begin{CodeInput}
fatdiag(1:27, steps=3)
\end{CodeInput}
\begin{CodeOutput}
      [,1] [,2] [,3] [,4] [,5] [,6] [,7] [,8] [,9]
 [1,]    1    4    7    0    0    0    0    0    0
 [2,]    2    5    8    0    0    0    0    0    0
 [3,]    3    6    9    0    0    0    0    0    0
 [4,]    0    0    0   10   13   16    0    0    0
 [5,]    0    0    0   11   14   17    0    0    0
 [6,]    0    0    0   12   15   18    0    0    0
 [7,]    0    0    0    0    0    0   19   22   25
 [8,]    0    0    0    0    0    0   20   23   26
 [9,]    0    0    0    0    0    0   21   24   27
\end{CodeOutput}
\end{CodeChunk}

We can extract a fat diagonal and diagonalise it again.

\begin{CodeChunk}
\begin{CodeInput}
m <- matrix(801:881, nrow=9, ncol=9)
fatdiag( fatdiag(m, steps=3), steps=3)
\end{CodeInput}
\begin{CodeOutput}
      [,1] [,2] [,3] [,4] [,5] [,6] [,7] [,8] [,9]
 [1,]  801  810  819    0    0    0    0    0    0
 [2,]  802  811  820    0    0    0    0    0    0
 [3,]  803  812  821    0    0    0    0    0    0
 [4,]    0    0    0  831  840  849    0    0    0
 [5,]    0    0    0  832  841  850    0    0    0
 [6,]    0    0    0  833  842  851    0    0    0
 [7,]    0    0    0    0    0    0  861  870  879
 [8,]    0    0    0    0    0    0  862  871  880
 [9,]    0    0    0    0    0    0  863  872  881
\end{CodeOutput}
\end{CodeChunk}

\section{Conclusion}\label{conclusion}

Higher-order arrays can sometimes be mapped to a matrix, which enables
us to visualise these arrays in a intuitive manner. However, the
standard matrix manipulations relating to diagonals become more complex
when we do so. The \texttt{diagonals} package provides the
\texttt{fatdiag()} function family, which enables the manipulation of
fat diagonals in \texttt{R}, using a syntax that is very close to the
\texttt{diag()} function family from \texttt{R}s \texttt{base} package.

\end{document}

