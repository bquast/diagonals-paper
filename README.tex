\documentclass[article]{jss}
\usepackage[utf8]{inputenc}

\author{
Bastiaan Quast\\The Graduate Institute, Geneva
}
\title{\pkg{diagonals}: Fat Diagonals in
R\thanks{This research was financed by the Swiss National Science Foundation (SNSF) under the grant 'Development Aid and Social Dynamics' (100018-140745) administered by the Graduate Institute's Centre on Conflict, Development and Peacebuilding (CCDP) and led by Jean-Louis Arcand. We thank Sandra Reimann and Oliver Jutersonke at the CCDP for their generous support.}}
\Keywords{matrix, diagonal, block, \proglang{R}}

\Abstract{
The diagonals package implements functions for handling fat diagonal
matrices, such as those that occur when multiple dimensions are mapped
along one or both edges of a matrix. The functions are called
\texttt{fatdiag} and \texttt{fatdiag\textless{}-} and are designed to
resemble the behaviour of the \texttt{base} package's \texttt{diag} and
\texttt{diag\textless{}-} functions.
}

\Plainauthor{Bastiaan Quast}
\Plaintitle{diagonals: Fat Diagonals in R}
\Shorttitle{\pkg{diagonals}: Fat Diagonals in R}
\Plainkeywords{matrix, diagonal, block, R}

%% publication information
%% \Volume{50}
%% \Issue{9}
%% \Month{June}
%% \Year{2012}
\Submitdate{}
%% \Acceptdate{2012-06-04}

\Address{
    Bastiaan Quast\\
  The Graduate Institute, Geneva\\
  Maison de la paix Geneva, Switzerland\\
  E-mail: \href{mailto:bquast@gmail.com}{\nolinkurl{bquast@gmail.com}}\\
  URL: \url{http://qua.st/}\\~\\
  }

\usepackage{amsmath}

\begin{document}

\section{Introduction}\label{introduction}

Diagonals are an important matrix manipulation. We present the
\texttt{diagonals} R package, which implements functions for dealing
with \textbf{fat diagonals}. Fat diagonals are block matrix-diagonals
that occur when two or more dimensions are mapped along a single edge of
a matrix. For an asymmetric network graph (e.g.~a dyadic social network)
to be mapped to a matrix, we would need each node along each edge of
matrix, however we would also need to map the direction of the tie,
which is an additional dimension. Typically these would be represented
as higher-order arrays (i.e. \texttt{length(dim(m))\textgreater{}=3}).
In order to effectively visualise such arrays, it can be helpful to do
so in a matrix (i.e. \texttt{length(dim(m))==2}). This could for
instance be represented as in the following matrix (where the \texttt{i}
and \texttt{o} suffices represent incoming and outgoing respectively).

\begin{CodeChunk}
\begin{CodeOutput}
  Ai Ao Bi Bo Ci Co Di Do
A  1  1  0  0  1  0  1  0
B  0  1  1  1  0  1  0  1
C  1  0  0  0  1  1  0  0
D  1  0  1  0  1  1  1  1
\end{CodeOutput}
\end{CodeChunk}

Sometimes the ties of a node to itself are not particularly meaningful
(e.g.~feeling of amiability towards oneself) and can be removed. For a
symmetric network this can simply be done using the function
\texttt{diag()} in R's \texttt{base} package, e.g.

\begin{CodeChunk}
\begin{CodeInput}
sm <- matrix(1, nrow=4, ncol=4,
             dimnames = list(c("A","B","C","D"),c("A","B","C","D")))
diag(sm) <- NA
sm
\end{CodeInput}
\begin{CodeOutput}
   A  B  C  D
A NA  1  1  1
B  1 NA  1  1
C  1  1 NA  1
D  1  1  1 NA
\end{CodeOutput}
\end{CodeChunk}

However, for higher-order matrices this does not work well, since the
diagonal follows the shortest dimension.

\begin{CodeChunk}
\begin{CodeInput}
diag(m) <- NA
m
\end{CodeInput}
\begin{CodeOutput}
  Ai Ao Bi Bo Ci Co Di Do
A NA  1  0  0  1  0  1  0
B  0 NA  1  1  0  1  0  1
C  1  0 NA  0  1  1  0  0
D  1  0  1 NA  1  1  1  1
\end{CodeOutput}
\end{CodeChunk}

In comes the \texttt{diagonals} package and its workhorse
\texttt{fatdiag()} function. The function is designed to mimic the
behaviour of the \texttt{diag()} as closely as possible, but with then
for \textbf{fat diagonals}.

\begin{CodeChunk}
\begin{CodeInput}
library(diagonals)
\end{CodeInput}
\end{CodeChunk}

\begin{CodeChunk}
\begin{CodeInput}
# the matrix m was restored to its original state
fatdiag(m, steps=4) <- NA
m
\end{CodeInput}
\begin{CodeOutput}
  Ai Ao Bi Bo Ci Co Di Do
A NA NA  0  0  1  0  1  0
B  0  1 NA NA  0  1  0  1
C  1  0  0  0 NA NA  0  0
D  1  0  1  0  1  1 NA NA
\end{CodeOutput}
\end{CodeChunk}

Note that the \texttt{steps} argument defines the number of steps on the
diagonal ladder. Alternatively we could set the \texttt{size} of the
step, more on this later.

The functions is this package where originally written in order to
support the \texttt{gvc} package, which implements a collection of
trade-flow indicators.

In trade flows use Inter-Country Input Output tables (ICIOs), which map
every \texttt{country * industry} combination along both edges of the
matrix. These tables are used to compute the Leontief inverse (Leontief
1936). In order to make the value of the Leontief internally comparable
we can normalise the inverse using post multiplication. Generally, this
post multiplication is done using a country's own exports, final demand,
etc. and since the \texttt{country * industry} dimensions are on each
edge, this gives us a higher-order matrix for which we want to extrac
the diagonal.

Existing packages that deal with similar topics \texttt{R}'s (R Core
Team 2015) package \texttt{Matrix} (Bates and Maechler 2015) and
specifically, for block diagonals the \texttt{bdsmatrix} package
(Therneau 2014) and the \texttt{jointDiag} package (Gouy-Pailler 2009).

\section{Data}\label{data}

As mentioned in the introduction, the typical use case is the mapping of
a 3- or 4-dimensional array to a matrix with two edges, this can be done
by combining two dimensions along one edge. This form of representation
occupies a middle group between the mathematically efficient
\texttt{array}, and the intuitive tidy data (see Wickham 2014), being
the most efficient way that is directly representable on paper.

The matrix used in the introduction is an example of a mapping of a
3-dimensional array.

\begin{CodeChunk}
\begin{CodeOutput}
  Ai Ao Bi Bo Ci Co Di Do
A  1  1  0  0  1  0  1  0
B  0  1  1  1  0  1  0  1
C  1  0  0  0  1  1  0  0
D  1  0  1  0  1  1  1  1
\end{CodeOutput}
\end{CodeChunk}

The \textbf{fat diagonals} occur when we have such a matrix and want to
select the diagonal along two of these dimensions, and all elements
along the third dimension. In the above matrix for instance, we would
select the diagonal along the dimensions \texttt{A:D} and \texttt{A:D},
and all elements along the dimension \texttt{i:o}. The \texttt{size} of
the the block on the diagonal here take the values \texttt{c(1,n)} with
\texttt{n} being the \texttt{length} of the third dimension.

\begin{CodeChunk}
\begin{CodeOutput}
  Ai Ao Bi Bo Ci Co Di Do
A NA NA  0  0  1  0  1  0
B  0  1 NA NA  0  1  0  1
C  1  0  0  0 NA NA  0  0
D  1  0  1  0  1  1 NA NA
\end{CodeOutput}
\end{CodeChunk}

In the case of a 4-dimensional array we would combine 2 dimensions along
each of the edges of the new matrix. An oft encountered example is trade
flow register called the Inter Country Input-Output table (ICIO). ICIOs
map all trade flows from countries and industries to each other. These
tables are generally represented as matrices where each industry for
each country is present along each edge, whereby rows represent outputs
and columns represent inputs. In order to compute certain trade
indicators for countries, it is often necessary to either extract the
fat diagonal, or zero it out. Implementing these indicators for the
\texttt{gvc} R package was the original impertus for implementing the
procedures which became the \texttt{diagonals} package.

An example of a minitiature ICIO is this, which \texttt{CH} represent
Switzerland and \texttt{RoW} represents the rest of the world, each have
three industries named \texttt{1}, \texttt{2}, and \texttt{3}.

\begin{CodeChunk}
\begin{CodeOutput}
     CH1 CH2 CH3 RoW1 RoW2 RoW3
CH1    1   0   0    1    1    0
CH2    0   0   1    1    0    1
CH3    1   0   0    0    0    0
RoW1   1   1   0    1    1    1
RoW2   1   0   1    1    0    0
RoW3   1   1   0    1    1    1
\end{CodeOutput}
\end{CodeChunk}

These fat diagonal matrices can be thought of as a generalised version
of a block diagonal matrix (see Rowland and Weisstein 2007). However,
although it is common for the combinations of dimensions along the edges
to be identicatal, and hence the \texttt{steps} square, this is not
necessary.

\section{Design}\label{design}

The implementation of fat diagonals in the \texttt{diagonals} package is
intended to be as close as possible to the functions dealing with
diagonals included in the \texttt{base} package. As such, the package
includes two functions.

\begin{itemize}
\itemsep1pt\parskip0pt\parsep0pt
\item
  \texttt{fatdiag()}
\item
  \texttt{fatdiag()\textless{}-}
\end{itemize}

These functions offer a very similar syntax to the base functions:

\begin{itemize}
\itemsep1pt\parskip0pt\parsep0pt
\item
  \texttt{diag()}
\item
  \texttt{diag()\textless{}-}
\end{itemize}

With the exception that the fat diagonal functions generally need more
information, in terms of the number of \texttt{steps} on the diagonal
ladder, or the \texttt{size} of these steps.

The function \texttt{fatdiag\textless{}-} like its base package
equivalent, replaces the (fat) diagonal of its first argument \texttt{x}
with the right side argument \texttt{value}. The \texttt{value} argument
can either be a scalar, in which case it is recycled for the length of
the diagonal, or it can be vector. For the base package function
\texttt{diag()\textless{}-} this vector has be of the same length as the
diagonal (here, the shortest dimension of the matrix), however, the
\texttt{fatdiag()\textless{}-} function will accept any vector that is
of a length that is an integer divisor of the length of the diagonal.
For example, if the length of the diagonal is \texttt{12}, then the
follow lengths for the replacement vector are accepted: \texttt{1},
\texttt{2}, \texttt{3}, \texttt{4}, and \texttt{6}.

The \texttt{fatdiag} function act similar to the \texttt{diag()}
function. Both these functions have two main applications . The first
application is (fat) diagonal extraction, is the first argument
\texttt{x} is a matrix, i.e. \texttt{length(dim(x)) == 2}, then the
function extracts the diagonal matrix and returns it as a vector.

The second application is (fat) diagonal matrix creation. This can be
done in two ways, using a scalar, or using a vector. If a scalar is used
for \texttt{x}, the \texttt{diag()} function returns an identity matrix
\texttt{Ix}, i.e.~a matrix of dimensions \texttt{x} times \texttt{x} is
returned, with \texttt{1} on the diagonal positions and \texttt{0}
elsewhere. The \texttt{fatdiag()} function supports the creation of
non-square matrices (e.g.~using \texttt{size = c(3,2)}) and therefore
uses \texttt{x} as the longest dimension of the matrix, where the other
dimension is determined automatically using the \texttt{size} argument.

In addition to the \texttt{fatdiag()} function family, a function to
create fat matrices from arrays is provided. The \texttt{matricise()}
function takes three- or four-dimensional array and outputs a matrix.
Resizing or transposing an array to a matrix is traditionally done using
\texttt{dim()\textless{}-} and \texttt{aperm} respectively, however
ordering in of the data using these methods does not create the fat
matrix design that we are looking for.

\section{Usage}\label{usage}

In the introduction we briefly demonstrate the usage of the
\texttt{fatdiag()} function for assigning a new \texttt{value} to the
fat diagonal. Here we take a closer look at some of the additional
functionality that is implemented.

\begin{CodeChunk}
\begin{CodeInput}
fatdiag(m, size=c(1,2) ) <- 881:888
m
\end{CodeInput}
\begin{CodeOutput}
   Ai  Ao  Bi  Bo  Ci  Co  Di  Do
A 881 882   0   0   1   0   1   0
B   0   1 883 884   0   1   0   1
C   1   0   0   0 885 886   0   0
D   1   0   1   0   1   1 887 888
\end{CodeOutput}
\end{CodeChunk}

So far we have been using the set \texttt{fatdiag()}, i.e.
\texttt{fatdiag()\textless{}-}. However, we can also use the
\texttt{fatdiag()} function either for diagonal extraction, or diagonal
matrix creation.

\begin{CodeChunk}
\begin{CodeInput}
fatdiag(m, steps = 4)
\end{CodeInput}
\begin{CodeOutput}
[1] 881 882 883 884 885 886 887 888
\end{CodeOutput}
\end{CodeChunk}

Fat diagonal matrices can be created using a scalar:

\begin{CodeChunk}
\begin{CodeInput}
fatdiag(6, steps=3)
\end{CodeInput}
\begin{CodeOutput}
     [,1] [,2] [,3] [,4] [,5] [,6]
[1,]    1    1    0    0    0    0
[2,]    1    1    0    0    0    0
[3,]    0    0    1    1    0    0
[4,]    0    0    1    1    0    0
[5,]    0    0    0    0    1    1
[6,]    0    0    0    0    1    1
\end{CodeOutput}
\end{CodeChunk}

or using a vector:

\begin{CodeChunk}
\begin{CodeInput}
fatdiag(1:12, steps=3)
\end{CodeInput}
\begin{CodeOutput}
     [,1] [,2] [,3] [,4] [,5] [,6]
[1,]    1    3    0    0    0    0
[2,]    2    4    0    0    0    0
[3,]    0    0    5    7    0    0
[4,]    0    0    6    8    0    0
[5,]    0    0    0    0    9   11
[6,]    0    0    0    0   10   12
\end{CodeOutput}
\end{CodeChunk}

We can extract a fat diagonal and diagonalise it again.

\begin{CodeChunk}
\begin{CodeInput}
m <- matrix(631:666, nrow=6, ncol=6)
(extr_fat_diag <- fatdiag(m, steps=3))
\end{CodeInput}
\begin{CodeOutput}
 [1] 631 632 637 638 645 646 651 652 659 660 665 666
\end{CodeOutput}
\begin{CodeInput}
fatdiag(extr_fat_diag , steps=3)
\end{CodeInput}
\begin{CodeOutput}
     [,1] [,2] [,3] [,4] [,5] [,6]
[1,]  631  637    0    0    0    0
[2,]  632  638    0    0    0    0
[3,]    0    0  645  651    0    0
[4,]    0    0  646  652    0    0
[5,]    0    0    0    0  659  665
[6,]    0    0    0    0  660  666
\end{CodeOutput}
\end{CodeChunk}

Note that the above code combines the two different ways in which the
\texttt{fatdiag()} function can be used, the first iteration extracts
the fat diagonal from the matrix \texttt{m} and returns it as a vector
\texttt{extr\_fat\_diag}, the second iteration takes the vector returned
by the first iteration and diagonalises it in a matrix, which is
returned.

The type of fat matrices that we analyse above can be created from
\texttt{array}s using the \texttt{matricise()} function.

\begin{CodeChunk}
\begin{CodeInput}
(a <- array( 1:81, dim = c(3,3,3,3) ) )
\end{CodeInput}
\begin{CodeOutput}
, , 1, 1

     [,1] [,2] [,3]
[1,]    1    4    7
[2,]    2    5    8
[3,]    3    6    9

, , 2, 1

     [,1] [,2] [,3]
[1,]   10   13   16
[2,]   11   14   17
[3,]   12   15   18

, , 3, 1

     [,1] [,2] [,3]
[1,]   19   22   25
[2,]   20   23   26
[3,]   21   24   27

, , 1, 2

     [,1] [,2] [,3]
[1,]   28   31   34
[2,]   29   32   35
[3,]   30   33   36

, , 2, 2

     [,1] [,2] [,3]
[1,]   37   40   43
[2,]   38   41   44
[3,]   39   42   45

, , 3, 2

     [,1] [,2] [,3]
[1,]   46   49   52
[2,]   47   50   53
[3,]   48   51   54

, , 1, 3

     [,1] [,2] [,3]
[1,]   55   58   61
[2,]   56   59   62
[3,]   57   60   63

, , 2, 3

     [,1] [,2] [,3]
[1,]   64   67   70
[2,]   65   68   71
[3,]   66   69   72

, , 3, 3

     [,1] [,2] [,3]
[1,]   73   76   79
[2,]   74   77   80
[3,]   75   78   81
\end{CodeOutput}
\end{CodeChunk}

\begin{CodeChunk}
\begin{CodeInput}
matricise(a, row_dim = 3, col_dim=4)
\end{CodeInput}
\begin{CodeOutput}
      [,1] [,2] [,3] [,4] [,5] [,6] [,7] [,8] [,9]
 [1,]    1    4    7   28   31   34   55   58   61
 [2,]    2    5    8   29   32   35   56   59   62
 [3,]    3    6    9   30   33   36   57   60   63
 [4,]   10   13   16   37   40   43   64   67   70
 [5,]   11   14   17   38   41   44   65   68   71
 [6,]   12   15   18   39   42   45   66   69   72
 [7,]   19   22   25   46   49   52   73   76   79
 [8,]   20   23   26   47   50   53   74   77   80
 [9,]   21   24   27   48   51   54   75   78   81
\end{CodeOutput}
\end{CodeChunk}

alternatively

\begin{CodeChunk}
\begin{CodeInput}
matricise(a, row_dim = 4, col_dim=3)
\end{CodeInput}
\begin{CodeOutput}
      [,1] [,2] [,3] [,4] [,5] [,6] [,7] [,8] [,9]
 [1,]    1    4    7   10   13   16   19   22   25
 [2,]    2    5    8   11   14   17   20   23   26
 [3,]    3    6    9   12   15   18   21   24   27
 [4,]   28   31   34   37   40   43   46   49   52
 [5,]   29   32   35   38   41   44   47   50   53
 [6,]   30   33   36   39   42   45   48   51   54
 [7,]   55   58   61   64   67   70   73   76   79
 [8,]   56   59   62   65   68   71   74   77   80
 [9,]   57   60   63   66   69   72   75   78   81
\end{CodeOutput}
\end{CodeChunk}

\section{Conclusion}\label{conclusion}

Higher-order arrays can sometimes be mapped to a matrix, which enables
us to visualise these arrays in a intuitive manner. However, the
standard matrix manipulations relating to diagonals become more complex
when we do so. The \texttt{diagonals} package provides the
\texttt{fatdiag()} function family, which enables the manipulation of
fat diagonals in \texttt{R}, using a syntax that is very close to the
\texttt{diag()} function family from \texttt{R}s \texttt{base} package.

\section*{References}\label{references}
\addcontentsline{toc}{section}{References}

Bates, Douglas, and Martin Maechler. 2015. \emph{Matrix: Sparse and
Dense Matrix Classes and Methods}.
\url{http://CRAN.R-project.org/package=Matrix}.

Gouy-Pailler, Cedric. 2009. \emph{JointDiag: Joint Approximate
Diagonalization of a Set of Square Matrices}.
\url{http://CRAN.R-project.org/package=jointDiag}.

Leontief, Wassily W. 1936. ``Quantitative Input and Output Relations in
the Economic Systems of the United States.'' \emph{The Review of
Economic Statistics}. JSTOR, 105--25.

R Core Team. 2015. \emph{R: A Language and Environment for Statistical
Computing}. Vienna, Austria: R Foundation for Statistical Computing.
\url{http://www.R-project.org/}.

Rowland, Todd, and Eric Weisstein. 2007. ``Block Diagonal Matrix.''
MathWorld. \url{http://mathworld.wolfram.com/BlockDiagonalMatrix.html}.

Therneau, Terry. 2014. \emph{Bdsmatrix: Routines for Block Diagonal
Symmetric Matrices}. \url{http://CRAN.R-project.org/package=bdsmatrix}.

Wickham, Hadley. 2014. ``Tidy Data.'' \emph{The Journal of Statistical
Software} 59 (10). \url{http://www.jstatsoft.org/v59/i10/}.

\end{document}

